% Options for packages loaded elsewhere
\PassOptionsToPackage{unicode}{hyperref}
\PassOptionsToPackage{hyphens}{url}
%
\documentclass[
]{book}
\usepackage{amsmath,amssymb}
\usepackage{lmodern}
\usepackage{iftex}
\ifPDFTeX
  \usepackage[T1]{fontenc}
  \usepackage[utf8]{inputenc}
  \usepackage{textcomp} % provide euro and other symbols
\else % if luatex or xetex
  \usepackage{unicode-math}
  \defaultfontfeatures{Scale=MatchLowercase}
  \defaultfontfeatures[\rmfamily]{Ligatures=TeX,Scale=1}
\fi
% Use upquote if available, for straight quotes in verbatim environments
\IfFileExists{upquote.sty}{\usepackage{upquote}}{}
\IfFileExists{microtype.sty}{% use microtype if available
  \usepackage[]{microtype}
  \UseMicrotypeSet[protrusion]{basicmath} % disable protrusion for tt fonts
}{}
\makeatletter
\@ifundefined{KOMAClassName}{% if non-KOMA class
  \IfFileExists{parskip.sty}{%
    \usepackage{parskip}
  }{% else
    \setlength{\parindent}{0pt}
    \setlength{\parskip}{6pt plus 2pt minus 1pt}}
}{% if KOMA class
  \KOMAoptions{parskip=half}}
\makeatother
\usepackage{xcolor}
\usepackage{longtable,booktabs,array}
\usepackage{calc} % for calculating minipage widths
% Correct order of tables after \paragraph or \subparagraph
\usepackage{etoolbox}
\makeatletter
\patchcmd\longtable{\par}{\if@noskipsec\mbox{}\fi\par}{}{}
\makeatother
% Allow footnotes in longtable head/foot
\IfFileExists{footnotehyper.sty}{\usepackage{footnotehyper}}{\usepackage{footnote}}
\makesavenoteenv{longtable}
\usepackage{graphicx}
\makeatletter
\def\maxwidth{\ifdim\Gin@nat@width>\linewidth\linewidth\else\Gin@nat@width\fi}
\def\maxheight{\ifdim\Gin@nat@height>\textheight\textheight\else\Gin@nat@height\fi}
\makeatother
% Scale images if necessary, so that they will not overflow the page
% margins by default, and it is still possible to overwrite the defaults
% using explicit options in \includegraphics[width, height, ...]{}
\setkeys{Gin}{width=\maxwidth,height=\maxheight,keepaspectratio}
% Set default figure placement to htbp
\makeatletter
\def\fps@figure{htbp}
\makeatother
\setlength{\emergencystretch}{3em} % prevent overfull lines
\providecommand{\tightlist}{%
  \setlength{\itemsep}{0pt}\setlength{\parskip}{0pt}}
\setcounter{secnumdepth}{5}
\usepackage{booktabs}
\ifLuaTeX
  \usepackage{selnolig}  % disable illegal ligatures
\fi
\usepackage[]{natbib}
\bibliographystyle{plainnat}
\IfFileExists{bookmark.sty}{\usepackage{bookmark}}{\usepackage{hyperref}}
\IfFileExists{xurl.sty}{\usepackage{xurl}}{} % add URL line breaks if available
\urlstyle{same} % disable monospaced font for URLs
\hypersetup{
  pdftitle={Elektrikernas kokbok},
  pdfauthor={Björn Ögren},
  hidelinks,
  pdfcreator={LaTeX via pandoc}}

\title{Elektrikernas kokbok}
\author{Björn Ögren}
\date{2022-11-10}

\begin{document}
\maketitle

{
\setcounter{tocdepth}{1}
\tableofcontents
}
\hypertarget{likstruxf6mskretsar}{%
\chapter{Likströmskretsar}\label{likstruxf6mskretsar}}

\[ f = 26,3 \ Hz  \]
\[ f = 26,3 \ Hz  \]
\[ f = 26,3 \ Hz  \]

\hypertarget{uri}{%
\section{URI}\label{uri}}

Hello World

\hypertarget{vuxe4xelstruxf6mskretsar}{%
\chapter{Växelströmskretsar}\label{vuxe4xelstruxf6mskretsar}}

Hello World

\hypertarget{tidsintervall}{%
\section{Tidsintervall}\label{tidsintervall}}

Frekvens är en storhet för antalet
repeterande händelser inom ett givet
tidsintervall{[}1{]}. För att beräkna
frekvensen fixerar man ett
tidsintervall, räknar antalet
förekomster av händelsen och
dividerar detta antal med längden av
tidsintervallet. Resultatet anges i
enheten hertz (Hz) efter den tyske
fysikern Heinrich Rudolf Hertz, där
1 Hz är en händelse som inträffar en
gång per sekund. Alternativt kan man
mäta tiden mellan två förekomster av
händelsen ((tids)perioden) och
därefter beräkna frekvensens
reciproka värde.

\hypertarget{frekvens}{%
\subsection{Frekvens}\label{frekvens}}

\begin{longtable}[]{@{}
  >{\raggedright\arraybackslash}p{(\columnwidth - 8\tabcolsep) * \real{0.1429}}
  >{\centering\arraybackslash}p{(\columnwidth - 8\tabcolsep) * \real{0.2143}}
  >{\centering\arraybackslash}p{(\columnwidth - 8\tabcolsep) * \real{0.2143}}
  >{\centering\arraybackslash}p{(\columnwidth - 8\tabcolsep) * \real{0.2143}}
  >{\centering\arraybackslash}p{(\columnwidth - 8\tabcolsep) * \real{0.2143}}@{}}
\toprule()
\begin{minipage}[b]{\linewidth}\raggedright
Samband
\end{minipage} & \begin{minipage}[b]{\linewidth}\centering
Beteckning
\end{minipage} & \begin{minipage}[b]{\linewidth}\centering
Storhet
\end{minipage} & \begin{minipage}[b]{\linewidth}\centering
Enhet
\end{minipage} & \begin{minipage}[b]{\linewidth}\centering
Förkortning
\end{minipage} \\
\midrule()
\endhead
\( Frekvens = \frac{1}{Tid} \) & \( f \) & Frekvens & Hertz & \( Hz \) \\
\bottomrule()
\end{longtable}

\begin{longtable}[]{@{}c@{}}
\toprule()
Exempel uträkning frekvens (1) \\
\midrule()
\endhead
\( Frekvens = \frac{1}{Tid} \) \\
\( f =\frac{1}{38} \times 10^{3} \) \\
\( f = 26,3 \ Hz  \) \\
\bottomrule()
\end{longtable}

\hypertarget{tid}{%
\subsection{Tid}\label{tid}}

\begin{longtable}[]{@{}
  >{\raggedright\arraybackslash}p{(\columnwidth - 8\tabcolsep) * \real{0.1429}}
  >{\centering\arraybackslash}p{(\columnwidth - 8\tabcolsep) * \real{0.2143}}
  >{\centering\arraybackslash}p{(\columnwidth - 8\tabcolsep) * \real{0.2143}}
  >{\centering\arraybackslash}p{(\columnwidth - 8\tabcolsep) * \real{0.2143}}
  >{\centering\arraybackslash}p{(\columnwidth - 8\tabcolsep) * \real{0.2143}}@{}}
\toprule()
\begin{minipage}[b]{\linewidth}\raggedright
Samband
\end{minipage} & \begin{minipage}[b]{\linewidth}\centering
Beteckning
\end{minipage} & \begin{minipage}[b]{\linewidth}\centering
Storhet
\end{minipage} & \begin{minipage}[b]{\linewidth}\centering
Enhet
\end{minipage} & \begin{minipage}[b]{\linewidth}\centering
Förkortning
\end{minipage} \\
\midrule()
\endhead
\( Tid = \frac{1}{Frekvens} \) & \( f \) & Frekvens & Hertz & \( Hz \) \\
\bottomrule()
\end{longtable}

\begin{longtable}[]{@{}c@{}}
\toprule()
Exempel uträkning tid (1) \\
\midrule()
\endhead
\( Tid = \frac{1}{Frekvens} \) \\
\( Tid = \frac{1}{400} \times 10^{3} \) \\
\( T = 2,5 \ ms  \) \\
\bottomrule()
\end{longtable}

\hypertarget{toppvuxe4rden}{%
\section{Toppvärden}\label{toppvuxe4rden}}

\hypertarget{toppspuxe4nning}{%
\subsection{Toppspänning}\label{toppspuxe4nning}}

\begin{longtable}[]{@{}
  >{\raggedright\arraybackslash}p{(\columnwidth - 8\tabcolsep) * \real{0.1429}}
  >{\centering\arraybackslash}p{(\columnwidth - 8\tabcolsep) * \real{0.2143}}
  >{\centering\arraybackslash}p{(\columnwidth - 8\tabcolsep) * \real{0.2143}}
  >{\centering\arraybackslash}p{(\columnwidth - 8\tabcolsep) * \real{0.2143}}
  >{\centering\arraybackslash}p{(\columnwidth - 8\tabcolsep) * \real{0.2143}}@{}}
\toprule()
\begin{minipage}[b]{\linewidth}\raggedright
Samband
\end{minipage} & \begin{minipage}[b]{\linewidth}\centering
Beteckning
\end{minipage} & \begin{minipage}[b]{\linewidth}\centering
Storhet
\end{minipage} & \begin{minipage}[b]{\linewidth}\centering
Enhet
\end{minipage} & \begin{minipage}[b]{\linewidth}\centering
Förkortning
\end{minipage} \\
\midrule()
\endhead
\( \widehat{u} = U_{eff} \times \sqrt{2} \) & \( \widehat{u} \) & Toppspänning & Volt & \( V \) \\
\bottomrule()
\end{longtable}

\begin{longtable}[]{@{}c@{}}
\toprule()
Exempel uträkning toppspänning \\
\midrule()
\endhead
\( \widehat{u} = U_{eff} \times \sqrt{2} \) \\
\( \widehat{u} = 415 \times \sqrt{2} \) \\
\( \widehat{u} \approx 587 \ V \) \\
\bottomrule()
\end{longtable}

\hypertarget{toppstruxf6m}{%
\subsection{Toppström}\label{toppstruxf6m}}

\begin{longtable}[]{@{}
  >{\raggedright\arraybackslash}p{(\columnwidth - 8\tabcolsep) * \real{0.1429}}
  >{\centering\arraybackslash}p{(\columnwidth - 8\tabcolsep) * \real{0.2143}}
  >{\centering\arraybackslash}p{(\columnwidth - 8\tabcolsep) * \real{0.2143}}
  >{\centering\arraybackslash}p{(\columnwidth - 8\tabcolsep) * \real{0.2143}}
  >{\centering\arraybackslash}p{(\columnwidth - 8\tabcolsep) * \real{0.2143}}@{}}
\toprule()
\begin{minipage}[b]{\linewidth}\raggedright
Samband
\end{minipage} & \begin{minipage}[b]{\linewidth}\centering
Beteckning
\end{minipage} & \begin{minipage}[b]{\linewidth}\centering
Storhet
\end{minipage} & \begin{minipage}[b]{\linewidth}\centering
Enhet
\end{minipage} & \begin{minipage}[b]{\linewidth}\centering
Förkortning
\end{minipage} \\
\midrule()
\endhead
\( \widehat{I} = I_{eff} \times \sqrt{2} \) & \( \widehat{I} \) & Toppström & Amper & \( A \) \\
\bottomrule()
\end{longtable}

\begin{longtable}[]{@{}c@{}}
\toprule()
Exempel uträkning toppström \\
\midrule()
\endhead
\( \widehat{I} = I_{eff} \times \sqrt{2} \) \\
\( \widehat{I} = 20 \times \sqrt{2} \) \\
\( \widehat{I} \approx 28,3 \ A  \) \\
\bottomrule()
\end{longtable}

\hypertarget{topp-till-toppspuxe4nning}{%
\subsection{Topp till toppspänning}\label{topp-till-toppspuxe4nning}}

\begin{longtable}[]{@{}
  >{\raggedright\arraybackslash}p{(\columnwidth - 8\tabcolsep) * \real{0.1429}}
  >{\centering\arraybackslash}p{(\columnwidth - 8\tabcolsep) * \real{0.2143}}
  >{\centering\arraybackslash}p{(\columnwidth - 8\tabcolsep) * \real{0.2143}}
  >{\centering\arraybackslash}p{(\columnwidth - 8\tabcolsep) * \real{0.2143}}
  >{\centering\arraybackslash}p{(\columnwidth - 8\tabcolsep) * \real{0.2143}}@{}}
\toprule()
\begin{minipage}[b]{\linewidth}\raggedright
Samband
\end{minipage} & \begin{minipage}[b]{\linewidth}\centering
Beteckning
\end{minipage} & \begin{minipage}[b]{\linewidth}\centering
Storhet
\end{minipage} & \begin{minipage}[b]{\linewidth}\centering
Enhet
\end{minipage} & \begin{minipage}[b]{\linewidth}\centering
Förkortning
\end{minipage} \\
\midrule()
\endhead
\( \widehat{\breve{u}} = \widehat{u} \times 2  \) & \( \widehat{\breve{u}} \) & Toppspänning & Volt & \( V \) \\
\bottomrule()
\end{longtable}

\begin{longtable}[]{@{}c@{}}
\toprule()
Exempel uträkning topp till toppspänning \\
\midrule()
\endhead
\( \widehat{\breve{u}} \) \\
\( \widehat{\breve{u}} = 587 \times 2 \) \\
\( \widehat{\breve{u}} = 1174 \ V \) \\
\bottomrule()
\end{longtable}

\hypertarget{topp-till-toppvuxe4rde-av-struxf6m}{%
\subsection{Topp till toppvärde av ström}\label{topp-till-toppvuxe4rde-av-struxf6m}}

\[ \widehat{\breve{I}} = \widehat{I} \times 2 \]

Exemple

\[ \widehat{\breve{I}} = \widehat{I} \times 2 \approx 28,3 \times 2 = 56 \ A \]

Topp till toppvärd är således
\[ \widehat{\breve{I}} \approx 56 \ A \]

\hypertarget{spuxe4nning}{%
\chapter{Spänning}\label{spuxe4nning}}

\hypertarget{y-koppling}{%
\section{Y-Koppling}\label{y-koppling}}

\hypertarget{linjespuxe4nning}{%
\subsection{Linjespänning}\label{linjespuxe4nning}}

\begin{longtable}[]{@{}
  >{\raggedright\arraybackslash}p{(\columnwidth - 8\tabcolsep) * \real{0.1429}}
  >{\centering\arraybackslash}p{(\columnwidth - 8\tabcolsep) * \real{0.2143}}
  >{\centering\arraybackslash}p{(\columnwidth - 8\tabcolsep) * \real{0.2143}}
  >{\centering\arraybackslash}p{(\columnwidth - 8\tabcolsep) * \real{0.2143}}
  >{\centering\arraybackslash}p{(\columnwidth - 8\tabcolsep) * \real{0.2143}}@{}}
\toprule()
\begin{minipage}[b]{\linewidth}\raggedright
Samband
\end{minipage} & \begin{minipage}[b]{\linewidth}\centering
Beteckning
\end{minipage} & \begin{minipage}[b]{\linewidth}\centering
Storhet
\end{minipage} & \begin{minipage}[b]{\linewidth}\centering
Enhet
\end{minipage} & \begin{minipage}[b]{\linewidth}\centering
Förkortning
\end{minipage} \\
\midrule()
\endhead
\( U_L = U_f \times \sqrt{3} \) & \( U_l \) & Spänning & Volt & \( V \) \\
\bottomrule()
\end{longtable}

\begin{longtable}[]{@{}c@{}}
\toprule()
Exempel uträkning fasspänning \\
\midrule()
\endhead
\( U_L = U_f \times \sqrt{3}  \) \\
\( U_L = 230 \times \sqrt{3} \) \\
\( U_L = 400 \ V \) \\
\bottomrule()
\end{longtable}

\hypertarget{fasspuxe4nning}{%
\subsection{Fasspänning}\label{fasspuxe4nning}}

\begin{longtable}[]{@{}
  >{\raggedright\arraybackslash}p{(\columnwidth - 8\tabcolsep) * \real{0.1429}}
  >{\centering\arraybackslash}p{(\columnwidth - 8\tabcolsep) * \real{0.2143}}
  >{\centering\arraybackslash}p{(\columnwidth - 8\tabcolsep) * \real{0.2143}}
  >{\centering\arraybackslash}p{(\columnwidth - 8\tabcolsep) * \real{0.2143}}
  >{\centering\arraybackslash}p{(\columnwidth - 8\tabcolsep) * \real{0.2143}}@{}}
\toprule()
\begin{minipage}[b]{\linewidth}\raggedright
Samband
\end{minipage} & \begin{minipage}[b]{\linewidth}\centering
Beteckning
\end{minipage} & \begin{minipage}[b]{\linewidth}\centering
Storhet
\end{minipage} & \begin{minipage}[b]{\linewidth}\centering
Enhet
\end{minipage} & \begin{minipage}[b]{\linewidth}\centering
Förkortning
\end{minipage} \\
\midrule()
\endhead
\( U_f = \frac{U_L}{\sqrt{3}} \) & \( U_l \) & Spänning & Volt & \( V \) \\
\bottomrule()
\end{longtable}

\begin{longtable}[]{@{}c@{}}
\toprule()
Exempel uträkning fasspänning \\
\midrule()
\endhead
\( U_f = \frac{U_L}{\sqrt{3}}  \) \\
\( U_f = \frac{400}{\sqrt{3}} \) \\
\( U_f = 230 \ V \) \\
\bottomrule()
\end{longtable}

\hypertarget{d-koppling}{%
\section{D-koppling}\label{d-koppling}}

\hypertarget{linjespuxe4nning-1}{%
\subsection{Linjespänning}\label{linjespuxe4nning-1}}

\hypertarget{fasspuxe4nning-1}{%
\subsection{Fasspänning}\label{fasspuxe4nning-1}}

\begin{longtable}[]{@{}
  >{\raggedright\arraybackslash}p{(\columnwidth - 8\tabcolsep) * \real{0.1429}}
  >{\centering\arraybackslash}p{(\columnwidth - 8\tabcolsep) * \real{0.2143}}
  >{\centering\arraybackslash}p{(\columnwidth - 8\tabcolsep) * \real{0.2143}}
  >{\centering\arraybackslash}p{(\columnwidth - 8\tabcolsep) * \real{0.2143}}
  >{\centering\arraybackslash}p{(\columnwidth - 8\tabcolsep) * \real{0.2143}}@{}}
\toprule()
\begin{minipage}[b]{\linewidth}\raggedright
Samband
\end{minipage} & \begin{minipage}[b]{\linewidth}\centering
Beteckning
\end{minipage} & \begin{minipage}[b]{\linewidth}\centering
Storhet
\end{minipage} & \begin{minipage}[b]{\linewidth}\centering
Enhet
\end{minipage} & \begin{minipage}[b]{\linewidth}\centering
Förkortning
\end{minipage} \\
\midrule()
\endhead
\( U_f = U_L = 400V  \) & \( U_L \) & Spänning & Volt & \( V \) \\
\bottomrule()
\end{longtable}

\hypertarget{struxf6m}{%
\chapter{Ström}\label{struxf6m}}

\hypertarget{y-koppling-1}{%
\section{Y-Koppling}\label{y-koppling-1}}

\hypertarget{fasstruxf6m}{%
\subsection{Fasström}\label{fasstruxf6m}}

\begin{longtable}[]{@{}
  >{\raggedright\arraybackslash}p{(\columnwidth - 8\tabcolsep) * \real{0.1429}}
  >{\centering\arraybackslash}p{(\columnwidth - 8\tabcolsep) * \real{0.2143}}
  >{\centering\arraybackslash}p{(\columnwidth - 8\tabcolsep) * \real{0.2143}}
  >{\centering\arraybackslash}p{(\columnwidth - 8\tabcolsep) * \real{0.2143}}
  >{\centering\arraybackslash}p{(\columnwidth - 8\tabcolsep) * \real{0.2143}}@{}}
\toprule()
\begin{minipage}[b]{\linewidth}\raggedright
Samband
\end{minipage} & \begin{minipage}[b]{\linewidth}\centering
Beteckning
\end{minipage} & \begin{minipage}[b]{\linewidth}\centering
Storhet
\end{minipage} & \begin{minipage}[b]{\linewidth}\centering
Enhet
\end{minipage} & \begin{minipage}[b]{\linewidth}\centering
Förkortning
\end{minipage} \\
\midrule()
\endhead
\( I_f = \frac{U_f}{R} \) & \( I_f \) & Ström & Ampere & \( A \) \\
\bottomrule()
\end{longtable}

\begin{longtable}[]{@{}c@{}}
\toprule()
Exempel uträkning fasström \\
\midrule()
\endhead
\( I_f = \frac{U_f}{R} \) \\
\( I_f \frac{400}{100} \) \\
\( I_f = 4 \ A \) \\
\bottomrule()
\end{longtable}

\hypertarget{linjestruxf6m}{%
\subsection{Linjeström}\label{linjestruxf6m}}

\[ I_L = I_f = Faström \]

\hypertarget{d-koppling-1}{%
\section{D-koppling}\label{d-koppling-1}}

\hypertarget{fasstruxf6m-1}{%
\subsection{Fasström}\label{fasstruxf6m-1}}

\begin{longtable}[]{@{}
  >{\raggedright\arraybackslash}p{(\columnwidth - 8\tabcolsep) * \real{0.1429}}
  >{\centering\arraybackslash}p{(\columnwidth - 8\tabcolsep) * \real{0.2143}}
  >{\centering\arraybackslash}p{(\columnwidth - 8\tabcolsep) * \real{0.2143}}
  >{\centering\arraybackslash}p{(\columnwidth - 8\tabcolsep) * \real{0.2143}}
  >{\centering\arraybackslash}p{(\columnwidth - 8\tabcolsep) * \real{0.2143}}@{}}
\toprule()
\begin{minipage}[b]{\linewidth}\raggedright
Samband
\end{minipage} & \begin{minipage}[b]{\linewidth}\centering
Beteckning
\end{minipage} & \begin{minipage}[b]{\linewidth}\centering
Storhet
\end{minipage} & \begin{minipage}[b]{\linewidth}\centering
Enhet
\end{minipage} & \begin{minipage}[b]{\linewidth}\centering
Förkortning
\end{minipage} \\
\midrule()
\endhead
\( I_f = \frac{U_h}{R} \) & \( I_f \) & Ström & Ampere & \( A \) \\
\bottomrule()
\end{longtable}

\begin{longtable}[]{@{}c@{}}
\toprule()
Exempel uträkning fasström \\
\midrule()
\endhead
\( I_f = \frac{U_h}{R} \) \\
\( I_f \frac{400}{100} \) \\
\( I_f = 4 \ A \) \\
\bottomrule()
\end{longtable}

\hypertarget{linjestruxf6m-1}{%
\subsection{Linjeström}\label{linjestruxf6m-1}}

\begin{longtable}[]{@{}
  >{\raggedright\arraybackslash}p{(\columnwidth - 8\tabcolsep) * \real{0.1429}}
  >{\centering\arraybackslash}p{(\columnwidth - 8\tabcolsep) * \real{0.2143}}
  >{\centering\arraybackslash}p{(\columnwidth - 8\tabcolsep) * \real{0.2143}}
  >{\centering\arraybackslash}p{(\columnwidth - 8\tabcolsep) * \real{0.2143}}
  >{\centering\arraybackslash}p{(\columnwidth - 8\tabcolsep) * \real{0.2143}}@{}}
\toprule()
\begin{minipage}[b]{\linewidth}\raggedright
Samband
\end{minipage} & \begin{minipage}[b]{\linewidth}\centering
Beteckning
\end{minipage} & \begin{minipage}[b]{\linewidth}\centering
Storhet
\end{minipage} & \begin{minipage}[b]{\linewidth}\centering
Enhet
\end{minipage} & \begin{minipage}[b]{\linewidth}\centering
Förkortning
\end{minipage} \\
\midrule()
\endhead
\( I_L = I_f \times \sqrt{3} \) & \( I_L \) & Ström & Ampere & \( A \) \\
\( I_L = \frac {P}{ \sqrt{3} \times U_h } \) & \( I_L \) & Ström & Ampere & \( A \) \\
\bottomrule()
\end{longtable}

\begin{longtable}[]{@{}c@{}}
\toprule()
Exempel uträkning linjeström (1) \\
\midrule()
\endhead
\( I_L = I_f \times \sqrt{3} \) \\
\( I_L = 90 \times \sqrt{3} \) \\
\( I_L = 2,3 \ A \) \\
\bottomrule()
\end{longtable}

\begin{longtable}[]{@{}c@{}}
\toprule()
Exempel uträkning linjeström (2) \\
\midrule()
\endhead
\( I_L = \frac {P}{ \sqrt{3} \times U_h} \) \\
\( I_L = \frac {6000}{ \sqrt{3} \times 400} \) \\
\( I_L = 8,7 \ A \) \\
\bottomrule()
\end{longtable}

\hypertarget{effekt}{%
\chapter{Effekt}\label{effekt}}

\hypertarget{trefaskretsar}{%
\section{Trefaskretsar}\label{trefaskretsar}}

Det finns en formel för beräkning av effekt och strömmar i trefaskretsar som gäller både för Y- och D-koppling. I praktiken är vi oftast intresserade av strömmarna som går i ledarna till en belastning, det vi kallar huvudström. Men i en D-koppling är det fasströmmarna genom belastningen som ger effektutvecklingen. Därför komplettear vi effektformeln med: \[ \sqrt{3} \] som beskriver sambandet mellan huvudström och fasström. Formeln utgör även grunden för beräkningar av effekten i reaktiva belastningar och den kompletteras då med \[ cos \phi \].

\begin{longtable}[]{@{}
  >{\raggedright\arraybackslash}p{(\columnwidth - 8\tabcolsep) * \real{0.1429}}
  >{\centering\arraybackslash}p{(\columnwidth - 8\tabcolsep) * \real{0.2143}}
  >{\centering\arraybackslash}p{(\columnwidth - 8\tabcolsep) * \real{0.2143}}
  >{\centering\arraybackslash}p{(\columnwidth - 8\tabcolsep) * \real{0.2143}}
  >{\centering\arraybackslash}p{(\columnwidth - 8\tabcolsep) * \real{0.2143}}@{}}
\toprule()
\begin{minipage}[b]{\linewidth}\raggedright
Samband
\end{minipage} & \begin{minipage}[b]{\linewidth}\centering
Beteckning
\end{minipage} & \begin{minipage}[b]{\linewidth}\centering
Storhet
\end{minipage} & \begin{minipage}[b]{\linewidth}\centering
Enhet
\end{minipage} & \begin{minipage}[b]{\linewidth}\centering
Förkortning
\end{minipage} \\
\midrule()
\endhead
\( P_{trefas}= \sqrt{3} \times U \times I_f \) & \( P \) & Aktiv effekt & Watt & \( W \) \\
\bottomrule()
\end{longtable}

\begin{longtable}[]{@{}c@{}}
\toprule()
Effekt i tre D-kopplade resistorer \\
\midrule()
\endhead
\( P_{trefas}= \sqrt{3} \times U \times I_f \) \\
\( P_{trefas}= \sqrt{3} \times U \times I_f \) \\
\( P= 4800 \ W \) \\
\bottomrule()
\end{longtable}

\hypertarget{reaktiva-kretsar}{%
\section{Reaktiva kretsar}\label{reaktiva-kretsar}}

\hypertarget{aktiv}{%
\subsection{Aktiv}\label{aktiv}}

Det är den aktivs effekt som vi kan omsätta till ljus, värme eller mekansik rörelse. Aktiva effekten har enheten watt och betecknas med P i effektriangeln.

\begin{longtable}[]{@{}
  >{\raggedright\arraybackslash}p{(\columnwidth - 8\tabcolsep) * \real{0.1429}}
  >{\centering\arraybackslash}p{(\columnwidth - 8\tabcolsep) * \real{0.2143}}
  >{\centering\arraybackslash}p{(\columnwidth - 8\tabcolsep) * \real{0.2143}}
  >{\centering\arraybackslash}p{(\columnwidth - 8\tabcolsep) * \real{0.2143}}
  >{\centering\arraybackslash}p{(\columnwidth - 8\tabcolsep) * \real{0.2143}}@{}}
\toprule()
\begin{minipage}[b]{\linewidth}\raggedright
Samband
\end{minipage} & \begin{minipage}[b]{\linewidth}\centering
Beteckning
\end{minipage} & \begin{minipage}[b]{\linewidth}\centering
Storhet
\end{minipage} & \begin{minipage}[b]{\linewidth}\centering
Enhet
\end{minipage} & \begin{minipage}[b]{\linewidth}\centering
Förkortning
\end{minipage} \\
\midrule()
\endhead
\( P= U \times I \times cos  \phi \) & \( P \) & Aktiv effekt & Watt & \( W \) \\
\( P_{trefas}= \sqrt{3} \times U \times I \times cos  \phi \) & \( P \) & Aktiv effekt & Watt & \( W \) \\
\bottomrule()
\end{longtable}

\begin{longtable}[]{@{}c@{}}
\toprule()
Exempel uträkning aktiv effekt \\
\midrule()
\endhead
\( P=U \times I \times cos  \phi \) \\
\( P=230 \times 0,78 \times 0,78 \) \\
\( P=1640 \ W \) \\
\bottomrule()
\end{longtable}

\begin{longtable}[]{@{}c@{}}
\toprule()
Exempel uträkning aktiv effekt trefas \\
\midrule()
\endhead
\( P_{trefas}= \sqrt{3} \times U \times I \times cos  \phi \) \\
\( P_{trefas}= \sqrt{3} \times ? \times ? \times ? \) \\
\( P_{trefas}= \ W \) \\
\bottomrule()
\end{longtable}

\hypertarget{skenbar}{%
\subsection{Skenbar}\label{skenbar}}

Skenbar effekt är produkten av strömmens och spänningens effektvärden. Skenbar effekt har enheten voltampere (VA).

\begin{longtable}[]{@{}
  >{\raggedright\arraybackslash}p{(\columnwidth - 8\tabcolsep) * \real{0.1429}}
  >{\centering\arraybackslash}p{(\columnwidth - 8\tabcolsep) * \real{0.2143}}
  >{\centering\arraybackslash}p{(\columnwidth - 8\tabcolsep) * \real{0.2143}}
  >{\centering\arraybackslash}p{(\columnwidth - 8\tabcolsep) * \real{0.2143}}
  >{\centering\arraybackslash}p{(\columnwidth - 8\tabcolsep) * \real{0.2143}}@{}}
\toprule()
\begin{minipage}[b]{\linewidth}\raggedright
Samband
\end{minipage} & \begin{minipage}[b]{\linewidth}\centering
Beteckning
\end{minipage} & \begin{minipage}[b]{\linewidth}\centering
Storhet
\end{minipage} & \begin{minipage}[b]{\linewidth}\centering
Enhet
\end{minipage} & \begin{minipage}[b]{\linewidth}\centering
Förkortning
\end{minipage} \\
\midrule()
\endhead
\( S=U \times I = \sqrt{P^2 + Q^2} \) & \( S \) & Skenbar effekt & Voltampere & \( VA \) \\
\( S_{trefas}= \sqrt{3} \times U \times I \) & \( S \) & Skenbar effekt & Voltampere & \( VA \) \\
\bottomrule()
\end{longtable}

\begin{longtable}[]{@{}c@{}}
\toprule()
Exempel uträkning skenbar effekt (1) \\
\midrule()
\endhead
\( S=U \times I \) \\
\( S=230 \times 9,05 \) \\
\( S=2081 \ W \) \\
\bottomrule()
\end{longtable}

\begin{longtable}[]{@{}c@{}}
\toprule()
Exempel uträkning skenbar effekt (2) \\
\midrule()
\endhead
\( S= \sqrt{P^2 + Q^2} \) \\
\( S= \sqrt{2000^2 + 1000^2} \) \\
\( S=2,2 \ kVA \) \\
\bottomrule()
\end{longtable}

\begin{longtable}[]{@{}c@{}}
\toprule()
Exempel uträkning skenbar effekt trefas \\
\midrule()
\endhead
\( S_{trefas}= \sqrt{3} \times U \times I \) \\
\( S_{trefas}= \sqrt{3} \times 230 \times 9,05 \) \\
\( S_{trefas}=2081 \ W \) \\
\bottomrule()
\end{longtable}

\hypertarget{reaktiv}{%
\subsection{Reaktiv}\label{reaktiv}}

Den reaktiva effekten uppstår på grund av fasförskjutningen som det reaktiva motståndet åstakomer. Den reaktiva effekten har enheten voltampere, VAr. Tillläget r står för reaktiv.

\begin{longtable}[]{@{}
  >{\raggedright\arraybackslash}p{(\columnwidth - 8\tabcolsep) * \real{0.1429}}
  >{\centering\arraybackslash}p{(\columnwidth - 8\tabcolsep) * \real{0.2143}}
  >{\centering\arraybackslash}p{(\columnwidth - 8\tabcolsep) * \real{0.2143}}
  >{\centering\arraybackslash}p{(\columnwidth - 8\tabcolsep) * \real{0.2143}}
  >{\centering\arraybackslash}p{(\columnwidth - 8\tabcolsep) * \real{0.2143}}@{}}
\toprule()
\begin{minipage}[b]{\linewidth}\raggedright
Samband
\end{minipage} & \begin{minipage}[b]{\linewidth}\centering
Beteckning
\end{minipage} & \begin{minipage}[b]{\linewidth}\centering
Storhet
\end{minipage} & \begin{minipage}[b]{\linewidth}\centering
Enhet
\end{minipage} & \begin{minipage}[b]{\linewidth}\centering
Förkortning
\end{minipage} \\
\midrule()
\endhead
\( Q=U \times I \times sin  \phi = \sqrt{S^{2}-P^{2}} \) & \( Q \) & Reaktiv effekt & Voltampere reakt & \( VAr \) \\
\bottomrule()
\end{longtable}

\begin{longtable}[]{@{}c@{}}
\toprule()
Exempel uträkning reaktiv effekt (1) \\
\midrule()
\endhead
\( Q=U \times I \times sin  \phi \) \\
\( Q=U \times I \times sin  \phi \) \\
\( Q=  \ VAr \) \\
\bottomrule()
\end{longtable}

\begin{longtable}[]{@{}c@{}}
\toprule()
Exempel uträkning reaktiv effekt (2) \\
\midrule()
\endhead
\( Q= \sqrt{S^{2}-P^{2}} \) \\
\( Q= \sqrt{1000^{2}-607^{2}} \) \\
\( Q= 795 \ VAr \) \\
\bottomrule()
\end{longtable}

\hypertarget{vuxe4xelstruxf6msmotstuxe5nd}{%
\chapter{Växelströmsmotstånd}\label{vuxe4xelstruxf6msmotstuxe5nd}}

\hypertarget{impedans}{%
\section{Impedans}\label{impedans}}

\begin{longtable}[]{@{}
  >{\raggedright\arraybackslash}p{(\columnwidth - 8\tabcolsep) * \real{0.1429}}
  >{\centering\arraybackslash}p{(\columnwidth - 8\tabcolsep) * \real{0.2143}}
  >{\centering\arraybackslash}p{(\columnwidth - 8\tabcolsep) * \real{0.2143}}
  >{\centering\arraybackslash}p{(\columnwidth - 8\tabcolsep) * \real{0.2143}}
  >{\centering\arraybackslash}p{(\columnwidth - 8\tabcolsep) * \real{0.2143}}@{}}
\toprule()
\begin{minipage}[b]{\linewidth}\raggedright
Samband
\end{minipage} & \begin{minipage}[b]{\linewidth}\centering
Beteckning
\end{minipage} & \begin{minipage}[b]{\linewidth}\centering
Storhet
\end{minipage} & \begin{minipage}[b]{\linewidth}\centering
Enhet
\end{minipage} & \begin{minipage}[b]{\linewidth}\centering
Förkortning
\end{minipage} \\
\midrule()
\endhead
\( Z=\sqrt{R^2 + (X_{L} - X_{L})^2} \) & \( Z \) & Impedans & Ohm & \(  \Omega \) \\
\( R = \frac{U}{I}  \) & \( R \) & Resistans & Ohm & \(  \Omega \) \\
\( X_L = 2 \pi fL \) & \( X_L \) & Induktiv reaktans & Ohm & \(  \Omega \) \\
\( X_C =\frac{1} {2 \pi f C}  \) & \( X_C \) & kapacitiv reaktans & Ohm & \(  \Omega \) \\
\bottomrule()
\end{longtable}

\begin{longtable}[]{@{}c@{}}
\toprule()
Exempel uträkning impedans Z \\
\midrule()
\endhead
\( Z=\sqrt{R^2 + (X_{L})^2} \) \\
\( Z=\sqrt{20^2 + (15,7)^2} \) \\
\( Z= 25,4 \  \Omega \) \\
\bottomrule()
\end{longtable}

\hypertarget{kondensatorer}{%
\section{Kondensatorer}\label{kondensatorer}}

Kondensatorns förmåga att lagra elektrisk laddning kallas kapacitans, och betecknas C. Enheten för kapacitans är farad som betecknas F.

\begin{longtable}[]{@{}lcc@{}}
\toprule()
Prefixer & Förkortning & Tiopotens \\
\midrule()
\endhead
\( 1 \ mikrofarad  \) & \(  \mu F \) & \( 10^{-6} \) \\
\( 1 \ nanofarad  \) & \( nF \) & \( 10^{-9} \) \\
\( 1 \ picofarad  \) & \( pF \) & \( 10^{-12} \) \\
\bottomrule()
\end{longtable}

\hypertarget{kapacitans}{%
\subsection{Kapacitans}\label{kapacitans}}

Kapacitans beskriver hur mycket energi kondensatorn kan innehålla vid en viss spänning.

\begin{longtable}[]{@{}
  >{\raggedright\arraybackslash}p{(\columnwidth - 8\tabcolsep) * \real{0.1429}}
  >{\centering\arraybackslash}p{(\columnwidth - 8\tabcolsep) * \real{0.2143}}
  >{\centering\arraybackslash}p{(\columnwidth - 8\tabcolsep) * \real{0.2143}}
  >{\centering\arraybackslash}p{(\columnwidth - 8\tabcolsep) * \real{0.2143}}
  >{\centering\arraybackslash}p{(\columnwidth - 8\tabcolsep) * \real{0.2143}}@{}}
\toprule()
\begin{minipage}[b]{\linewidth}\raggedright
Samband
\end{minipage} & \begin{minipage}[b]{\linewidth}\centering
Beteckning
\end{minipage} & \begin{minipage}[b]{\linewidth}\centering
Storhet
\end{minipage} & \begin{minipage}[b]{\linewidth}\centering
Enhet
\end{minipage} & \begin{minipage}[b]{\linewidth}\centering
Förkortning
\end{minipage} \\
\midrule()
\endhead
\( C=\frac{1}{(2 \pi f  X_c)} \) & \( C \) & Kapacitans & Farad & \( F^{As/V} \) \\
\( f = \frac{1}{T}   \) & \( f \) & Hertz & \( Hz \) & \\
\( 2 \times \pi = 3.14  \) & \( Pi \) & Omkrets & Radies & \( \pi \) \\
\bottomrule()
\end{longtable}

\begin{longtable}[]{@{}
  >{\centering\arraybackslash}p{(\columnwidth - 0\tabcolsep) * \real{1.0000}}@{}}
\toprule()
\begin{minipage}[b]{\linewidth}\centering
Exempel uträkning kapacitans
\end{minipage} \\
\midrule()
\endhead
\( L=\frac{X_L} {2 \pi f} \) \\
\( L=\frac{1000} {(2 \times 3.14 \times 1.0 \times 10^{3} \ \sqrt{3})} \) \\
\( L= 0.16 \ H \) \\
\bottomrule()
\end{longtable}

\hypertarget{kapacitiv-reaktans}{%
\subsection{Kapacitiv reaktans}\label{kapacitiv-reaktans}}

Växelströmsmotståndet i kondensatorn minskar när frekvensen ökar. Då kommer ekvationen att minska när frekvesen ökar.

\begin{longtable}[]{@{}
  >{\raggedright\arraybackslash}p{(\columnwidth - 8\tabcolsep) * \real{0.1429}}
  >{\centering\arraybackslash}p{(\columnwidth - 8\tabcolsep) * \real{0.2143}}
  >{\centering\arraybackslash}p{(\columnwidth - 8\tabcolsep) * \real{0.2143}}
  >{\centering\arraybackslash}p{(\columnwidth - 8\tabcolsep) * \real{0.2143}}
  >{\centering\arraybackslash}p{(\columnwidth - 8\tabcolsep) * \real{0.2143}}@{}}
\toprule()
\begin{minipage}[b]{\linewidth}\raggedright
Samband
\end{minipage} & \begin{minipage}[b]{\linewidth}\centering
Beteckning
\end{minipage} & \begin{minipage}[b]{\linewidth}\centering
Storhet
\end{minipage} & \begin{minipage}[b]{\linewidth}\centering
Enhet
\end{minipage} & \begin{minipage}[b]{\linewidth}\centering
Förkortning
\end{minipage} \\
\midrule()
\endhead
\( X_C =\frac{1} {2 \pi f C}  \) & \( X_C \) & kapacitiv reaktans & Ohm & \(  \Omega \) \\
\( f = \frac{1}{T}   \) & \( f \) & Hertz & \( Hz \) & \\
\( 2 \times \pi = 3.14  \) & \( Pi \) & Omkrets & Radies & \( \pi \) \\
\bottomrule()
\end{longtable}

\begin{longtable}[]{@{}c@{}}
\toprule()
Exempel uträkning kapacitiv reaktans \\
\midrule()
\endhead
\( X_C =\frac{1} {2 \pi f C} \) \\
\( X_C = \frac{1}{2 \times \pi \times 50 \times 0,0002} \) \\
\( X_C = 15,91 \  \Omega  \) \\
\bottomrule()
\end{longtable}

\hypertarget{seriekopplade}{%
\subsection{Seriekopplade}\label{seriekopplade}}

\begin{longtable}[]{@{}
  >{\raggedright\arraybackslash}p{(\columnwidth - 8\tabcolsep) * \real{0.1429}}
  >{\centering\arraybackslash}p{(\columnwidth - 8\tabcolsep) * \real{0.2143}}
  >{\centering\arraybackslash}p{(\columnwidth - 8\tabcolsep) * \real{0.2143}}
  >{\centering\arraybackslash}p{(\columnwidth - 8\tabcolsep) * \real{0.2143}}
  >{\centering\arraybackslash}p{(\columnwidth - 8\tabcolsep) * \real{0.2143}}@{}}
\toprule()
\begin{minipage}[b]{\linewidth}\raggedright
Samband
\end{minipage} & \begin{minipage}[b]{\linewidth}\centering
Beteckning
\end{minipage} & \begin{minipage}[b]{\linewidth}\centering
Storhet
\end{minipage} & \begin{minipage}[b]{\linewidth}\centering
Enhet
\end{minipage} & \begin{minipage}[b]{\linewidth}\centering
Förkortning
\end{minipage} \\
\midrule()
\endhead
\( C_{tot}=C_{1} + C_{2}... \) & \( C \) & Kapacitans & Farad & \( F^{As/V} \) \\
\bottomrule()
\end{longtable}

\begin{longtable}[]{@{}c@{}}
\toprule()
Exempel uträkning kapacitiv reaktans \\
\midrule()
\endhead
\( C_{tot}=C_{1} + C_{2} \) \\
\( C_{tot}=12_{1} + 12_{2} \) \\
\( C_{tot}=24 \  \mu F \) \\
\bottomrule()
\end{longtable}

\hypertarget{parallellkopplade}{%
\subsection{Parallellkopplade}\label{parallellkopplade}}

\begin{longtable}[]{@{}
  >{\raggedright\arraybackslash}p{(\columnwidth - 8\tabcolsep) * \real{0.1429}}
  >{\centering\arraybackslash}p{(\columnwidth - 8\tabcolsep) * \real{0.2143}}
  >{\centering\arraybackslash}p{(\columnwidth - 8\tabcolsep) * \real{0.2143}}
  >{\centering\arraybackslash}p{(\columnwidth - 8\tabcolsep) * \real{0.2143}}
  >{\centering\arraybackslash}p{(\columnwidth - 8\tabcolsep) * \real{0.2143}}@{}}
\toprule()
\begin{minipage}[b]{\linewidth}\raggedright
Samband
\end{minipage} & \begin{minipage}[b]{\linewidth}\centering
Beteckning
\end{minipage} & \begin{minipage}[b]{\linewidth}\centering
Storhet
\end{minipage} & \begin{minipage}[b]{\linewidth}\centering
Enhet
\end{minipage} & \begin{minipage}[b]{\linewidth}\centering
Förkortning
\end{minipage} \\
\midrule()
\endhead
\( 
\frac{1}{C_{tot}} =
\frac{1}{C_{1}} +
\frac{1}{C_{2}} +
\frac{1}{C_{3}}
...
\) & \( C \) & Kapacitans & Farad & \( F^{As/V} \) \\
\bottomrule()
\end{longtable}

\begin{longtable}[]{@{}
  >{\centering\arraybackslash}p{(\columnwidth - 0\tabcolsep) * \real{1.0000}}@{}}
\toprule()
\begin{minipage}[b]{\linewidth}\centering
Exempel uträkning kapacitiv reaktans
\end{minipage} \\
\midrule()
\endhead
\( 
\frac{1}{C_{tot}} =
\frac{1}{C_{1}} +
\frac{1}{C_{2}} +
\frac{1}{C_{3}}
...
\) \\
\( 
\frac{1}{C_{tot}} =
\frac{1}{1,8_{1}} +
\frac{1}{16_{2}} +
\frac{1}{32_{3}}
\) \\
\( C_{tot}=4.5 \ nF \) \\
\bottomrule()
\end{longtable}

\hypertarget{spolar}{%
\section{Spolar}\label{spolar}}

Spolens egenskaper kallas induktans, betecknas i formler L och mäts i enheten Henry (H).

\begin{longtable}[]{@{}lcc@{}}
\toprule()
Prefixer & Enhet & Förkostning \\
\midrule()
\endhead
\( 1 \ millihenry  \) & \( mH \) & \( 10^{-3} \) \\
\( 1 \ mikrohenry  \) & \(  \mu H \) & \( 10^{-6} \) \\
\bottomrule()
\end{longtable}

\hypertarget{induktans}{%
\subsection{Induktans}\label{induktans}}

Induktansen beror på hur många varv spolen har, diametern, avståndet mellan ledarna och om spolen är försedd med järnkärna. Flera lindningsvarv och större diameter ger spolen större indutans.

\begin{longtable}[]{@{}
  >{\raggedright\arraybackslash}p{(\columnwidth - 8\tabcolsep) * \real{0.1429}}
  >{\centering\arraybackslash}p{(\columnwidth - 8\tabcolsep) * \real{0.2143}}
  >{\centering\arraybackslash}p{(\columnwidth - 8\tabcolsep) * \real{0.2143}}
  >{\centering\arraybackslash}p{(\columnwidth - 8\tabcolsep) * \real{0.2143}}
  >{\centering\arraybackslash}p{(\columnwidth - 8\tabcolsep) * \real{0.2143}}@{}}
\toprule()
\begin{minipage}[b]{\linewidth}\raggedright
Samband
\end{minipage} & \begin{minipage}[b]{\linewidth}\centering
Beteckning
\end{minipage} & \begin{minipage}[b]{\linewidth}\centering
Storhet
\end{minipage} & \begin{minipage}[b]{\linewidth}\centering
Enhet
\end{minipage} & \begin{minipage}[b]{\linewidth}\centering
Förkortning
\end{minipage} \\
\midrule()
\endhead
\( L=\frac{X_L} {2\pi f} \) & \( L \) & Induktans & Henry & \( H^{Vs/A} \) \\
\( f = \frac{1}{T}   \) & \( f \) & Frekvens & Hertz & \( Hz \) \\
\( 2 \times \pi = 3.14  \) & \( Pi \) & ? & ? & \( \pi \) \\
\bottomrule()
\end{longtable}

\[ L=\frac{X_L} {2 \times \pi f} \]
\[ L=\frac{1000} {(2 \times 3.14 \times 1.0 \times 10^{3} \ \sqrt{3})} \]
\[ L= 0.16 \ H \]

\hypertarget{induktiv-reaktans}{%
\subsection{Induktiv reaktans}\label{induktiv-reaktans}}

Växelströmsmotståndet är
frekvensberoende och motståndet ökar när
frekvensen ökar.

\begin{longtable}[]{@{}
  >{\raggedright\arraybackslash}p{(\columnwidth - 8\tabcolsep) * \real{0.1429}}
  >{\centering\arraybackslash}p{(\columnwidth - 8\tabcolsep) * \real{0.2143}}
  >{\centering\arraybackslash}p{(\columnwidth - 8\tabcolsep) * \real{0.2143}}
  >{\centering\arraybackslash}p{(\columnwidth - 8\tabcolsep) * \real{0.2143}}
  >{\centering\arraybackslash}p{(\columnwidth - 8\tabcolsep) * \real{0.2143}}@{}}
\toprule()
\begin{minipage}[b]{\linewidth}\raggedright
Samband
\end{minipage} & \begin{minipage}[b]{\linewidth}\centering
Beteckning
\end{minipage} & \begin{minipage}[b]{\linewidth}\centering
Storhet
\end{minipage} & \begin{minipage}[b]{\linewidth}\centering
Enhet
\end{minipage} & \begin{minipage}[b]{\linewidth}\centering
Förkortning
\end{minipage} \\
\midrule()
\endhead
\( X_L = 2 \pi fL \) & \( X_L \) & Induktiv reaktans & Ohm & \(  \Omega \) \\
\( f = \frac{1}{T} \) & \( f \) & Frekvens & Hertz & \( Hz \) \\
\( 2 \times \pi = 3.14  \) & \( Pi \) & ? & ? & \( \pi \) \\
\bottomrule()
\end{longtable}

\[ X_L = 2 \pi fL \]
\[ X_L = 2 \times \pi \ 50 \ Hz \times 0,05 \ H  \]
\[ X_L=15,7 \  \Omega \]

  \bibliography{book.bib,packages.bib}

\end{document}
